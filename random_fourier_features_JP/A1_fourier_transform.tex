% TeX source
%
% Author: Tetsuya Ishikawa <tiskw111@gmail.com>
% Date  : October 03, 2020
%%%%%%%%%%%%%%%%%%%%%%%%%%%%%%%%%%%%%%%%%%%%%%%%%%%%% SOURCE START %%%%%%%%%%%%%%%%%%%%%%%%%%%%%%%%%%%%%%%%%%%%%%%%%%%%%

本節では,定理証明の際に省略したFourier変換およびParsevalの等式の解説を行う.
本節でも数学的シンプルさを優先して周波数ベースのFourier変換を用いることとする.

\subsection*{Fourier変換に関する補足}

Fourier変換の定義としては本文の式(\ref{eqn:def_fourier_transform})で十分だが,
より良い表記,より深い理解のために,Fourier変換を関数空間の内積を用いて表現してみよう.
まず関数$\eta_{\xi}(x)$を
\begin{equation}
    \eta_{\xi}(x) \triangleq e^{2\pi i x \xi},
\end{equation}
と定義する.次に関数の内積$\langle \cdot, \cdot \rangle$を最も標準的な内積演算
\begin{equation}
    \langle f, g \rangle \triangleq \int_{-\infty}^{\infty} f(x) g^*(x) \, \mathrm{d}x,
\end{equation}
で定めるとしよう.ただし$g^*(x)$は関数$g(x)$の複素共役である.
このとき式(\ref{eqn:def_fourier_transform})で表現でされるFourier変換は
\begin{equation}
    \mathcal{F}[f](\xi) = \langle f, \eta_{\xi} \rangle,
\end{equation}
と表すことができる.
このとき関数群$\{ \eta_{\xi}(x) \,|\, \xi \in \mathbb{R} \}$は正規直交基底と非常に似た性質を持つ.すなわち
\begin{equation}
    \langle \eta_{\xi_1}, \eta_{\xi_2} \rangle = \delta(\xi_1 - \xi_2),
    \label{eqn:fourier_base}
\end{equation}
が成り立つ.ただし$\delta$はDiracのデルタ関数 (\textit{Dirac delta function}) である.
つまり式(\ref{eqn:fourier_base})は$\xi_1 = \xi_2$のときのみ値を持つ.
これは通常のベクトル空間の基底$\{ \bs{e}_1, \bs{e}_2, \ldots \}$が持つ性質
\begin{equation}
    \bs{e}_i\tran \bs{e}_j =
    \begin{cases}
        1, & i = j, \\
        0, & i \neq j,
    \end{cases}
\end{equation}
と非常に良く似ている.
したがってFourier変換は正規直交基底との内積,つまるところただの基底交換のようなもの,と見ることもできるのである.

\begin{displayquote}\footnotesize\textsf{NOTE:}
    Fourier変換の定義を初めて見た若かりし頃,諸君らも
    「何でFourier変換が逆回転$e^{-2\pi i x \xi}$で逆Fourier変換が順回転$e^{2\pi i x \xi}$なんだよ,フツーは逆だろ!」
    とキレかかった経験があるのではないだろうか.上述の内積によるFourier変換の理解はこの疑問を解消してくれる.
    そう,基底$\eta_{\xi}(x)$はちゃんと順回転$e^{2\pi i x \xi}$で定義されていたのだ.
    しかし内積を計算する際に共役をとる必要があり,そこで回転方向が反転したのである.少しはスッキリしたであろうか.
\end{displayquote}

\subsection*{Fourier変換の収束性について}

Parsevalの等式の証明に移る前に,一部の読者が気になっているであろう
Fourier変換(\ref{eqn:def_fourier_transform})の収束性について簡単に述べておこう.
つまり,式(\ref{eqn:def_fourier_transform})が定義できるのはどんな関数$f$か?という問題である.
結論から言えば,式(\ref{eqn:def_fourier_transform})が収束する条件は
\begin{equation}
    \int_{-\infty}^{\infty} |f(x)| \, \mathrm{d}x < \infty,
\end{equation}
である.ハッキリ言って,上式を満たす関数は応用上の観点から見ても非常に少ない.
例えば三角関数や指数関数ですら上式の制約は満たさない.
もちろん世の中には病的な関数が沢山あるゆえ,真に全ての関数に対してFourier変換を定義せよというのは困難であるのだが,
とはいえ,少なくとも応用上必要な関数に対してはFourier変換ができないと困ってしまう.
この問題に対するアプローチは様々あるが,著者の知る限りでは,
\begin{enumerate}
    \item[(1)] 扱う関数はすべて有限の台 (\textit{support}) を持つと仮定する方法.
               すなわち,関数が値を持つ範囲を十分に広い有限な範囲に限定してしまうという方法.
               関数の台が有限であれば,応用上必要な関数はほぼすべて積分可能となる.
    \item[(2)] 両側Laplace変換の特殊な場合 ($s = i\omega$) としてFourier変換を定義する方法である.
               実は両側Laplace変換は,解析接続を用いた巧妙な仕組みにより,非常に幅広い関数に対して定義可能である.
\end{enumerate}
の2つのアプローチがある.初学者は(1)の理解で良いであろう.
世の中の多くの人の理解は以下の(1)の方法で理解していると思われる.
ちなみに著者は理論的にも便利な(2)で理解している.興味があれば調べてみると良い.

\subsection*{Parsevalの等式}

次にParsevalの等式 (\textit{Parseval's identity}) について見ていこう.
一般にParsevalの等式と言うと$\|f\|^2 = \|\widehat{f}\|^2$を指すことも多いが,本文書ではより一般的な形の定理を証明する.
Fourier変換が正規直交基底を用いた基底の交換であることが分かれば,ほぼ自明な定理である.

\begin{theorem}[Parsevalの等式]
    任意のFourier変換可能な関数$f$, $g$およびそのFourier変換$\widehat{f}$, $\widehat{g}$に対して
    $\langle f, g \rangle = \langle \widehat{f}, \widehat{g} \rangle$が成り立つ.
\end{theorem}
\begin{proof}
    証明は直接計算による.
    \begin{align*}
        \langle \widehat{f}, \widehat{g} \rangle
        &= \int_{-\infty}^{-\infty} \langle f, \eta_{\xi} \rangle \langle g, 
        \eta_{\xi} \rangle \, \mathrm{d}\xi \\
        &= \iiint_{-\infty}^{-\infty}  f(x_1) g(x_2) \eta_{\xi}(x_1) \eta_{\xi}(x_2)
        \, \mathrm{d}x_1 \mathrm{d}x_2 \mathrm{d}\xi \\
        \intertext{ここで$\eta_{\xi}(x_1) \eta_{\xi}(x_2) = \eta_{x_1}(\xi) \eta_{x_2}(\xi)$に注意すれば}
        &= \iint_{-\infty}^{-\infty}  f(x_1) g(x_2)
        \langle \eta_{x_1}, \eta_{x_2} \rangle \, \mathrm{d}x_1 \mathrm{d}x_2 \\
        \intertext{関数$\eta_{x}(\xi)$の正規直交性に注意すれば
        $\langle \eta_{x_1}, \eta_{x_2} \rangle$は$\delta(x_1 - x_2)$であるから}
        &= \int_{-\infty}^{-\infty}  f(x) g(x) \, \mathrm{d}x
        = \langle f, g \rangle,
    \end{align*}
    となり,題意が成り立つ.
\end{proof}

ここまで読み進めた読者には指摘するまでもないであろうが,
上記定理の特殊な場合として$f = g$とおけば,良く知られた$\|f\|^2 = \|\widehat{f}\|^2$を得る.

%%%%%%%%%%%%%%%%%%%%%%%%%%%%%%%%%%%%%%%%%%%%%%%%%%%%% SOURCE FINISH %%%%%%%%%%%%%%%%%%%%%%%%%%%%%%%%%%%%%%%%%%%%%%%%%%%%
% vim: expandtab tabstop=4 shiftwidth=4 fdm=marker
