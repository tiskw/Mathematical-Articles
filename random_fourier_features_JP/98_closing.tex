% TeX source
%
% Author: Tetsuya Ishikawa <tiskw111@gmail.com>
% Date  : March 01, 2020
%%%%%%%%%%%%%%%%%%%%%%%%%%%%%%%%%%%%%%%%%%%%%%%%%%%%% SOURCE START %%%%%%%%%%%%%%%%%%%%%%%%%%%%%%%%%%%%%%%%%%%%%%%%%%%%%

本文書では,非常に簡単にではありますが,Random Fourier Featuresの解説を試みてみました.
個人的には,簡単のためにデータを1次元にして説明しようかとも考えたのですが,
さすがに実応用の段階でつまづくであろうと思い,最初から多次元のデータを想定して書いてみました.
この点についてはもしかしたら将来的に修正するかもしれません.

Random Fourier Featuresの拡張としては,2017年にGoogleよりORF (\textit{Orthogonal Random Features})
および SORF (\textit{Structured ORF}) が提案されています.
これはRFFのランダム行列$\bs{W}$が直行行列でないことに注目し,直交性を強制的に持たせることでより高い近似効率を狙ったものです.
私も手元で実装し,いくつかの有名なデータセットに対してORFおよびSORFを試しましたが,
確かにカーネルの近似精度は高いのですが,各データセットの推論精度という意味ではRFFと大きな差はありませんでした.
どうやらカーネルの近似精度と推論精度は単純には相関しないようです.
まだ仮説の域を出ませんが,おそらく有限次元近似が良い正則化として機能しているのでしょう.
本文書でORFやSORFに触れていないのは,それが主な理由です.

私の数学の師は常に
「人生で一度も証明を与えたことのない定理を自ら使用するのは数学的犯罪である」
と私に教えて下さいました.これは教育的な配慮だけでは決してないと思います.
内部を理解せずに扱うことによる誤用,それの招く惨めな結末.数学だけに限った話ではないでしょう.
近年の機械学習の研究者やエンジニアはこれをよくよく心に留めおくべきです.もちろん私自身を含めて,です.

最後に,Toyota Research Institute Advanced Development, Inc. の石原昌弘さんには
本文書の誤植を指摘して頂きました.この場を借りて厚く御礼申し上げます.
また,私の数学的活動は,2017年に逝去された恩師,山下弘一郎先生や,
大学および大学院で私の指導教官を担当して下さった早川朋久准教授をはじめ,
数学で私と関わりを持ったすべての方々のおかげで成り立っています.
そして,数学的活動の以前に,そもそも私の生は両親によって与えられ,妻によって支えられています.

%%%%%%%%%%%%%%%%%%%%%%%%%%%%%%%%%%%%%%%%%%%%%%%%%%%%% SOURCE FINISH %%%%%%%%%%%%%%%%%%%%%%%%%%%%%%%%%%%%%%%%%%%%%%%%%%%%
% vim: expandtab tabstop=4 shiftwidth=4 fdm=marker
