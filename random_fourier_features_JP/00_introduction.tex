% TeX source
%
% Author: Tetsuya Ishikawa <tiskw111@gmail.com>
% Date  : March 01, 2020
%%%%%%%%%%%%%%%%%%%%%%%%%%%%%%%%%%%%%%%%%%%%%%%%%%%%% SOURCE START %%%%%%%%%%%%%%%%%%%%%%%%%%%%%%%%%%%%%%%%%%%%%%%%%%%%%

本文書はRahimi~\textit{et al}.~\cite{Rahimi2007}によって2007年に提案された
Random Fourier Features(以下,RFFと略す)について,カーネル法をある程度ご存じの読者を対象に解説した文書です.
RFFは,端的に言えばカーネル法におけるカーネル関数を有限次元近似する手法であり,
カーネル法をより高速に,そして大規模なデータセットに適用することができるようになる手法です.

本文書では,記号の確認も含めてカーネル法について簡単にご説明はしていますが,
初めてカーネル法に触れる読者には不十分極まりないと思います.
カーネル法をご存じない方は,事前に赤穂~\cite{Akaho2008}やBishop~\cite{Bishop2010}などに
目を通して頂くことをお薦めします.

また,本文書は意欲的な高校生や大学生諸君にとっても有用と思います.
大学初年度の教養数学をきちんと修めていれば,本文書を理解することはさほど困難ではありません.
少し背伸びをすれば高校生諸君でも十分に理解できる内容です.
さらにRFFは,2007年に提案され,すでに2000件近くの引用がなされ,現在でも改良が続けられている手法です.
すでに10年以上昔の成果ですので,変化の激しい機械学習の分野ではもはや新しい結果とは言えません.
それでも学生の皆さんが普段勉強している内容と比較すれば極めて最近の,
そして現在でも研究対象として注目され続けている結果です.
大昔の結果ばかり勉強させられ,世界の最先端がどこにあるのか分からなくなってしまった方々にとって,
この文書は良いきっかけを提供できるかもしれません.

また,本文書の最後では,深層学習との関係性についても簡単に触れています.
というのも,本文書を書くきっかけは
\begin{center}
「カーネルサポートベクターマシンを深層学習のフレームワークで実装したいんだが,
どうすりゃ良いんだコノヤロー!」
\end{center}
という質問(挑発?)を職場の同僚から受けたことに端を発しているためです.
この問いもまたRFFを用いることで肯定的に解決できます.

何はともあれ,まずはRFFの目的と意義から順に見ていくことにしましょう.

\begin{displayquote}\footnotesize\textsf{NOTE:}
同僚の名誉のために述べておくと,上述の深層学習のフレームワークとは
TensorflowやPyTorchのような有名なワークステーション向けフレームワークではなく,組み込み向けのマイナーなものです.
このフレームワークは,ニューラルネットワークの実装を主目的に設計されていること,組み込み向けゆえにサポートされている機能が極めて限定的なことから,
カーネルサポートベクターマシンのようなニューラルネットワークでない機能を実装するのは容易ではないという事情があります.
TensorflowやPyTorchのような柔軟性の高いフレームワークであればRFFを用いずとも実現は簡単です.
\end{displayquote}

%%%%%%%%%%%%%%%%%%%%%%%%%%%%%%%%%%%%%%%%%%%%%%%%%%%%% SOURCE FINISH %%%%%%%%%%%%%%%%%%%%%%%%%%%%%%%%%%%%%%%%%%%%%%%%%%%%
% vim: expandtab tabstop=4 shiftwidth=4 fdm=marker
