% TeX source
%
% Author: Tetsuya Ishikawa <tiskw111@gmail.com>
% Date  : Aug  4, 2019
%%%%%%%%%%%%%%%%%%%%%%%%%%%%%%%%%%%% SOURCE START %%%%%%%%%%%%%%%%%%%%%%%%%%%%%%%%%%%

\subsection*{シュワルツの不等式}

関数空間$L^2$上の内積の性質としてシュワルツの不等式(\textit{Schwarz inequality})を紹介したが,
これ再掲し証明しよう.

\begin{theorem}[シュワルツの不等式]
関数空間$L^2$に対し,$f, g \in L^2$, $a, b \in \mathbb{R}$とする.
このとき
\begin{equation}
\langle f, g \rangle = \| f \| \, \| g \|,
\end{equation}
が成り立つ.
\end{theorem}
\begin{proof}
任意の$\alpha, \beta \in \mathbb{R}$に対して
$2 \alpha \beta \leq \alpha^2 + \beta^2$が成り立つことは明らかであるから,
$\alpha = f(x) / \| f \|$, $\beta = g(x) / \| g \|$とおけば
\begin{equation}
    \frac{2 f(x) g(x)}{\| f \| \, \| g \|}
    \leq \frac{f(x)^2}{\| f \|^2} + \frac{g(x)^2}{\| g \|^2},
\end{equation}
が成り立つ.上式の両辺を区間$(-\infty, \infty)$で積分すれば,
\begin{equation}
    \frac{2 \langle f, g \rangle}{\| f \| \, \| g \|}
    \leq \frac{\| f \|^2}{\| f \|^2} + \frac{\| g \|^2}{\| g \|^2} = 2,
\end{equation}
を得る.上式より示すべき式は直ちに導かれる.
\end{proof}


\subsection*{三角不等式}

関数空間$L^2$上のノルムの性質として三角不等式を紹介したが,これ再掲し証明しよう.

\begin{theorem}[三角不等式]
関数空間$L^2$に対し,$f, g \in L^2$とする.このとき
\begin{equation}
    \| f + g \| \leq \| f \| + \| g \|,
\end{equation}
が成り立つ.
\end{theorem}
\begin{proof}
シュワルツの不等式によれば
\begin{equation}
\| f + g \|^2
\leq \| f \|^2 + \langle f, g \rangle + \| g \|^2
= \left( \| f \|^2 + \| g \|^2 \right)^2,
\end{equation}
が成り立つ.よって証明すべきは明らか.
\end{proof}

%%%%%%%%%%%%%%%%%%%%%%%%%%%%%%%%%%%% SOURCE FINISH %%%%%%%%%%%%%%%%%%%%%%%%%%%%%%%%%%
% vim: expandtab shiftwidth=4 tabstop=4 filetype=tex
