% TeX source
%
% Author: Tetsuya Ishikawa <tiskw111@gmail.com>
% Date  : Aug  4, 2019
%%%%%%%%%%%%%%%%%%%%%%%%%%%%%%%%%%%% SOURCE START %%%%%%%%%%%%%%%%%%%%%%%%%%%%%%%%%%%

\subsection{カーネル回帰の定式化}

カーネル回帰の定式化が理解出来ていれば,K-SVMの定式化は容易である.
K-SVMは,一般化されたSVM(\ref{eqn:general_svm_1})-(\ref{eqn:general_svm_3})
における$f_w(\bs{x})$をカーネル回帰に置きかえたものであり,
結局のところ,最適化問題(\ref{eqn:general_svm_3})を解くことに帰着される.
まずは$f_w(\bs{x})$をカーネル回帰に置きかえた最適化問題(\ref{eqn:general_svm_3})を明らかにしよう.
式(\ref{eqn:kernel_reg_curve})を最適化問題(\ref{eqn:general_svm_3})に代入し,
さらに正則化項を付加すると
\begin{equation}\begin{split}
    \min_{\xi_i, a_i \in \mathbb{R}} \hspace{6pt}& \sum_{i \in \mathcal{I}} \xi_i
    + \lambda \sum_{i \in \mathcal{I}} \sum_{j \in \mathcal{I}} a_i k_{ij} a_j, \\
    \text{s.t.} \hspace{6pt}
    & \xi_i \geq 1 - y_i \sum_{j \in \mathcal{I}} k_{ij} a_j, \hspace{ 5pt} \text{for all $i \in \mathcal{I}$}, \\
    & \xi_i \geq 0,                                           \hspace{65pt} \text{for all $i \in \mathcal{I}$},
\end{split}\end{equation}
となり,さらにこれを行列形式にまとめると
\begin{equation}\begin{split}
    \label{eqn:ksvm}
    \min_{\bs{\xi}, \bs{a} \in \mathbb{R}^N} \hspace{6pt}& \bs{1}\tran \bs{\xi} + \lambda \bs{a}\tran \bs{Ka}, \\
    \text{s.t.} \hspace{6pt}
    & \bs{\xi} \geq \bs{1} - \bs{La}, \\
    & \bs{\xi} \geq \bs{0},
\end{split}\end{equation}
となる.ただし
\begin{align}
    \bs{\xi} \triangleq
    \begin{pmatrix}
        \xi_1 \\ \vdots \\ \xi_n
    \end{pmatrix},
    \hspace{7pt}
    \bs{L} \triangleq
    \begin{pmatrix}
        y_1 k_{11} & \cdots & y_1 k_{1n} \\
        \vdots     & \ddots & \vdots     \\
        y_n k_{n1} & \cdots & y_n k_{nn}
    \end{pmatrix}
\end{align}
である.
式(\ref{eqn:ksvm})は2次計画問題に帰着することができ,
かつ行列$\bs{K}$が正定値行列であることから(これはカーネル関数が正定値関数であることから直ちにしたがう),
最適解は唯一存在し,また数値的にも安定して解くことができる.


\subsection{カーネルサポートベクターマシンのPython実装}

K-SVMは先にも紹介したLIBSVMやscikit-learnから利用可能である.
特にscikit-learnは統一的なインタフェースで様々な機械学習アルゴリズムが利用可能となっており,
実応用上はこちらのパッケージを利用するのが良い.

%%%%%%%%%%%%%%%%%%%%%%%%%%%%%%%%%%%% SOURCE FINISH %%%%%%%%%%%%%%%%%%%%%%%%%%%%%%%%%%
% vim: expandtab shiftwidth=4 tabstop=4 filetype=tex
